% Une ligne commentaire débute par le caractère « % »

\documentclass[a4paper]{article}

% Options possibles : 10pt, 11pt, 12pt (taille de la fonte)
%                     oneside, twoside (recto simple, recto-verso)
%                     draft, final (stade de développement)

\usepackage[utf8]{inputenc}   % LaTeX, comprends les accents !
\usepackage[T1]{fontenc}      % Police contenant les caractères français
\usepackage[francais]{babel}  
\usepackage[colorinlistoftodos]{todonotes} 
\usepackage{pgfgantt} 
\usepackage{amsmath} 
\usepackage{listings}

\definecolor{dkgreen}{rgb}{0,0.6,0}
\definecolor{gray}{rgb}{0.5,0.5,0.5}
\definecolor{mauve}{rgb}{0.58,0,0.82}

\lstset{frame=tb,
  language=Java,
  aboveskip=3mm,
  belowskip=3mm,
  showstringspaces=false,
  columns=flexible,
  basicstyle={\small\ttfamily},
  numbers=none,
  numberstyle=\tiny\color{gray},
  keywordstyle=\color{blue},
  commentstyle=\color{dkgreen},
  stringstyle=\color{mauve},
  breaklines=true,
  breakatwhitespace=true,
  tabsize=3
}


\usepackage[a4paper,left=2cm,right=2cm]{geometry}% Format de la page, réduction des marges
\usepackage{graphicx}  % pour inclure des images

%\pagestyle{headings}        % Pour mettre des entêtes avec les titres
                              % des sections en haut de page


\begin{document}
\centerline{\Huge\bf HAI405I}
\vspace*{1.5cm}
\begin{center}               % pour centrer 
	
	\framebox[8cm]{
  %\includegraphics[width=10cm]{logo.pdf}   % insertion d'une image
	ici un logo si vous le souhaitez.
	}

\end{center}
\vspace*{1.5cm}

\fbox{\centerline{\Huge Projet de programmation}}

\vspace*{1.5cm}

\noindent{\large\bf Groupe 16 :}

\begin{itemize}
\item Roman Chris
\item Garcia Charly
\item Bonnafé Rémi
\item Hafdane Léo
\end{itemize}
\vspace*{1.5cm}
\begin{center}
  L2 informatique\\
  Faculté des Sciences\\
Université de Montpellier.
\end{center}

\newpage

\section{Introduction}
Le projet que nous avons mené durant ces deux semaines consiste a développer un site de jeux de cartes avec React, un framework JavaScript utilisé pour la création d'interfaces utilisateur interactives. Notre équipe, composée de Léo, Rémi, Chris, et récemment renforcée par l'arrivée de Charly, a donc conçu ce site.

Au cours de la première semaine, notre objectif principal était de mettre en place le site avec un jeux: la bataille. Nous avons commencé par définir le schéma visuel de chaque page, puis nous avons procédé à la mise en place du serveur. Les tâches étaient réparties entre les membres de l'équipe.

Dans la deuxième semaine, nous avons accueilli Charly au sein de notre équipe. Ensemble,  notre objectif etait l'ajout du jeux 6 qui prends et l'amelioration de diverses fonctionnalitées.

Dans ce rapport, nous présenterons l'architecture de notre projet, notre utilisation de React pour la création de composants réutilisables, ainsi que notre protocole de communication et d'échange de données. Enfin, nous dresserons un bilan de notre expérience et des accomplissements réalisés au cours de ces deux semaines de travail intense et collaboratif.


\section{Organisation}
Semaine 1 :

Le groupe a découvert react et a construit le schéma visuel de chaque pages 
puis à mis en place le serveur. 
	Léo s'est principalement occupé du serveur, notamment la mise en place des socket pour assurer la communication avec le client, Rémi s'occupait de la page « plateau bataille » et de ses composants : « carte », « joueurBataille », « start » et « monJeux », Chris a travaillé sur la page « créer/rejoindre » et ses composants ensuite Léo à mis en place la base de données pour gérer les comptes utilisateurs. Chris s'est occupé des fichiers png. Rémi et Chris se sont occupés de la mise en forme css et de la liaisons avec le serveur des pages existantes et ont aussi crée les composants : « chat » et « sélection jeux ». Tandis que Léo a créé les classes des jeux et a complété celle de la bataille.

Semaine 2:

	Charly a rejoint le groupe ! Il a découvert react et le projet.
Léo a créé la classe 6 qui prends pendant que Chris et Rémi ont fait la page associée et ses composants ; modification de : « carte » et  « chat »
	Charly a fait le tableau des scores en fin de partie , Rémi s’est occupé d enregister les scores et parties dans la base de donnée pour les utiliser dans le « profil » et le « leaderboard » créé avec Chris. Chris a créé l’audio et Léo a ; reglé des problèmes d historique, fait une navbar et créé des SVG très jolis pour le chat, la navbar et l'audio pour une meilleur experience utilisateur. 
Léo et Rémi ont fait un timer pour le choix de la carte ou de la colonne.Chris a créé un rank dans le profil.

\newpage
\definecolor{barblue}{RGB}{153,204,254}
\definecolor{linkred}{RGB}{165,0,33}

\begin{center}
\begin{ganttchart} [vgrid, hgrid] {1}{14}
  \gantttitle{organigramme du projet}{14} \\
  \gantttitlelist{1,...,14}{1} \\
\ganttgroup{Rémi} {1} {14} \\

\ganttbar{decouverte react} {1}{2} 
\ganttnewline
\ganttbar {develloppement des composants "bataille"} {2}{7} 
\ganttnewline
\ganttbar {mise en forme css} {3} {14} 
\ganttnewline
\ganttbar {mise en place des fichiers png} {8} {8} 
\ganttnewline
\ganttbar {develloppement des composants "6 qui prends"} {8} {14} 
\ganttnewline
\ganttgroup{Léo} {1} {14} \\

\ganttbar {decouverte react} {1}{2}  
\ganttnewline
\ganttbar {installation serveur} {2} {3} 
\ganttnewline
\ganttbar {creation de la base de données}{3}{5} 
\ganttnewline
\ganttbar {mise en forme css} {3} {14} 
\ganttnewline
\ganttbar {creation des classes jeux}{4}{7} 
\ganttnewline
\ganttbar {creation de la classe "6 qui prends"}{8}{9} 
\ganttnewline
\ganttbar {develloppement des composants globaux} {10} {14} 
\ganttnewline
\ganttgroup{Chris} {1} {14} \\

\ganttbar {decouverte react} {1}{2} 
\ganttnewline
\ganttbar {mise en place des fichiers png} {2} {2} 
\ganttnewline
\ganttbar {develloppement des composants "bataille"} {2} {7} 
\ganttnewline
\ganttbar {mise en forme css} {3} {14} 
\ganttnewline
\ganttbar {develloppement des composants "6 qui prends"} {8} {14} 
\ganttnewline
\ganttgroup{Charly} {1} {14} \\
\ganttbar {decouverte react} {8}{9} 
\ganttnewline
\ganttbar {creation du tableau fin de partie} {10}{14}
\end{ganttchart}
\end{center}

\section{Architecture, protocole de communication et échange de données}
Dans notre Projet, 3 grandes entitées communiquent:\\

-le Client  \\

-le Serveur React \\

-le Serveur Node \\

Nous avons vu que le serveur React sert à faire le rendu de composants React depuis un serveur Node qui l'envoi aux clients, cepandant, nous n'avons pas utilisé cette fonctionalité.

Dans notre projet, le serveur react et node communiquent avec le protocole websocket (nous n'utilisons pas http), websocket permet d'échanger des variables et listes de variables via une requête. Le protocole http pourrait être utilisé par exemple pour les "room", le lien de la partie pourrait être une l'objet d une requête http.\\ \\
Voici comment nous avons mis en place socket:
\begin{lstlisting}
const express = require("express");
const app = express();
const http = require("http");
const { Server } = require("socket.io");
const cors = require("cors");
app.use(cors());
const httpServer = http.createServer(app);
const io = new Server(httpServer, {
    cors: {
        origin: "*",
        methods: ["GET", "POST"],
    },
});
const port = process.env.HAI405I_PORT || 3001;
httpServer.listen(port, () => {
    console.log(`server running on port http://localhost:${port}/`);
});

const sockets = {}; // clef: socket.id              valeur: {compte, partie}
const parties = {}; // clef: code de la partie      valeur: instance de jeu

\end{lstlisting}
Voici une utilisation de socket:\\
émission de la requête:
\begin{lstlisting}
    function carteClick() {
        socket.emit("reqCoup", { carte: { valeur: valeur, type: type }, index: props.index });
    }
\end{lstlisting}
reception de la requête:
\begin{lstlisting}
    socket.on("reqCoup", async (json) => {
        if (!sockets[socket.id]) {
            return;
        }
        const carte = json.carte;
        const index = json.index;
        const code = sockets[socket.id].partie;
        const jeux = parties[code];
        if (!jeux || !jeux.coup(socket.id, carte, index)) {
            return;
        }
        resPlayers(code);
        resPlateau(code);
        if (jeux.everyonePlayed()) {
            if (jeux.nextRound()) {
                await new Promise((r) => setTimeout(r, 1000));
                resPlayers(code);
                resPlateau(code);
            }
        }
    });
\end{lstlisting}

\section{Utilisation de React}
React est un framwork qui permet la création de composants autonomes et réutilisables qui retournent du html. Ces composants peuvent être associés pour créer des pages complexes.

Cette architecture nous a permis par exemple de créer un composant carte que l on utilise dans toutes des pages "in game", nous pouvons aussi créer des composants avec des paramêtres :

\begin{lstlisting}

import "./Start.css";

function Start(props) {
    return (
        <div id="divStart">
            <label className="code">code de la partie:</label>
            <label className="code">{props.code}</label>
            <button
                hidden={!props.afficheStart}
                onClick={props.start}
            >
                commencer
            </button>
            <button
                hidden={!props.afficheSave}
                onClick={props.save}
            >
                save
            </button>
        </div>
    );
}
export default Start;

\end{lstlisting}

Ce composant peut être utilisé dans n'importe quelle autre composant ou page avec la balise :
\begin{lstlisting}

import Carte from "../../../component/Carte/Carte";

<Carte visible=...\>

\end{lstlisting}

\section{Bilan}
Au terme de ces deux semaines de travail sur notre projet nous pouvons constater quelques difficultés rencontrées ainsi qu un apprentissage en react.

Difficultés rencontrées :

Sauvegarde des parties : L'une des principales difficultés auxquelles nous avons été confrontés a été la mise en œuvre de la fonction de sauvegarde des parties. Nous avons rencontré des obstacles techniques.

Mise en place du serveur : Un autre défi majeur a été la configuration et la mise en place du serveur, en particulier dans le contexte de la communication entre le serveur React et le serveur Node.

Analyse rétrospective :

Sur le plan organisationnel, nous avons appris l'importance de GITHUB lors d'un projet complexe. La répartition claire des tâches, l'entre aide et la mise en place de points de contrôle réguliers, ont contribué à maintenir notre équipe sur la bonne voie malgré les défis rencontrés.

Du point de vue technique, nous avons appris a utiliser React et a configurer des serveurs.

En fin de compte ce projet nous a appris a travailler en groupe et a développer en react.

\end{document}
